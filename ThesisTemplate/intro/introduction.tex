\chapter{Introduction}
\pagenumbering{arabic} \setcounter{page}{1}

\section{Background}

With the evolution of machine learning and predictive modelling being used to forecast the outcome of sporting events, as well as analysing every moment of a particular game to improve team performance, find an opponent's weakness and improve tactics, could an artificial intelligence be trained to predict the outcome of a given race with reasonable accuracy? Predictive modelling techniques have previously been used in forecasting the outcome of sporting \cite{2015arXiv151105837K,trap2493,5647440}.
This project will research the various techniques used by several parties such as the teams themselves, bookmakers and spectators to see they can be used to predict the outcome of certain events. The research into how certain parties predict the outcome of the event is not the only research that will take place, research into data collection will be done, as well as the main body of the research which is to find a suitable set of algorithms to train the model on.

\section{Aim}
To investigate how various machine learning techniques could be utilised to create predictive models for the outcome of sporting events, with a particular focus on racing.


\section{Objectives}

\begin{enumerate}
    \item To review various techniques previously used to predict the outcome of sporting events and, more specifically, the results of races.
    \item To review appropriate evaluation techniques which can be used to assess the performance of the predictive techniques identified.
    \item To identify an appropriate method of data collection for creating a dataset which can be leveraged by the system for the purpose of training and testing.
    \item Implement a system which utilises appropriate machine learning techniques identified during the background research stage to build models for predicting the outcome of a given sporting event (e.g. a horse race).
    \item Evaluate the capabilities of the model via the techniques previously identified.
\end{enumerate}

\section{Product}
The main product at the end of this project will be the research seeing if a model can be trained to correctly predict the outcome of a race. A product will be developed, this research artefact is the model itself that the research led to being built, and also helped further the research into this topic. In this section, the product will be referred to as the research artefact. This is due to the research being the main focus of this project and not a final product. The main part of the research will be the choices of which algorithms will be used to train the model, to further the research into this topic.
The artefact will be the model that trains off of data collected on a set of races, after the model is trained it will be tested on another part of the data. This test data will be used to determine the accuracy of the model.
\section{Rationale}
Artificial intelligence and machine learning are having an impact on the way sports are analysed \cite{MIN2008551} . Specific groups are going to benefit from the research that is to take place in many different ways. Some of these ways are: Bookmakers may benefit from such research into racing due to there being a way that a horse can pay out more for placing than it can by winning \cite{levitt_2014}. If a horse is to pay out more for placing than winning, than bookmakers could leverage a system to tell them the possibility of both outcomes and adjust odds so they don't have to pay out as much. Another benefit of research surrounding not just racing but sports as a whole is having real time analytics fed into a decision-making system for coaches to the be given processed data to make decisions off of \cite{7029116}. Finally, another party benefiting from such a system that will be built are spectators of the sport, spectators could benefit in many ways, one of which is that they can know whether or not to place a bet on how likely the outcome is to happen.
\\ \\

\section{Chapter List}

A short summary of the chapters of this report.
\begin{itemize}
	\item \textbf{Chapter \ref{ch:Literature Review}} Literature Review. This chapter looks at previous works surrounding relevant topics to this project.
	\item \textbf{Chapter \ref{ch:Background Research}} Background Research. Research into topics concluded upon in the literature review.
	\item \textbf{Chapter \ref{ch:Design}} Design. Looking into the design on the artefact.
	\item \textbf{Chapter \ref{ch:Data}} Dataset. A summary upon the dataset used in this project. 
	\item \textbf{Chapter \ref{ch:Implementation}} Implementation of the artfact is discussed in this chapter.
	\item \textbf{Chapter \ref{ch:Evaluation}} Evaluation \& Testing. This chapter deals with the evaluation and testing of the artefact.
	\item \textbf{Chapter \ref{ch:Conclusion}} Conclusion. The conclusions of the report are presented.
\end{itemize}

